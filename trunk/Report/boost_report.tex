\documentclass[a4paper,twoside,12pt]{article}
% Alternative Options:
%	Paper Size: a4paper / a5paper / b5paper / letterpaper / legalpaper / executivepaper
% Duplex: oneside / twoside
% Base Font Size: 10pt / 11pt / 12pt


%% Language %%%%%%%%%%%%%%%%%%%%%%%%%%%%%%%%%%%%%%%%%%%%%%%%%
\usepackage[english,french]{babel}
\usepackage[latin1]{inputenc}
\usepackage[T1]{fontenc}
\usepackage{ae}

\usepackage{lmodern} %Type1-font for non-english texts and characters

\usepackage[top=2.5cm, bottom=2cm, left=2cm, right=2cm]{geometry}
\usepackage{icomma} % Permet l'utilisation de virgule comme s�parateur d�cimal
\usepackage{url} % Package pour ne pas avoir des probl�mes avec des URL's
% Utiliser \url{}

%% Packages for Graphics & Figures %%%%%%%%%%%%%%%%%%%%%%%%%%
%\usepackage[dvips]{graphicx}
%\usepackage{color,psfrag}
\usepackage{graphicx} %%For loading graphic files
%\usepackage{subfig} %%Subfigures inside a figure
%\usepackage{tikz} %%Generate vector graphics from within LaTeX
%\usepackage{tikz-3dplot} %requires 3dplot.sty to be in same directory, or in your LaTeX installation
%\usetikzlibrary{calc} %pour faire les calculs

%\usepackage{epstopdf}

\newcommand{\HRule}{\rule{\linewidth}{0.5mm}} %Dessine la ligne horizontale

%% Please note:
%% Images can be included using \includegraphics{filename}
%% resp. using the dialog in the Insert menu.
%% 
%% The mode "LaTeX => PDF" allows the following formats:
%%   .jpg  .png  .pdf  .mps
%% 
%% The modes "LaTeX => DVI", "LaTeX => PS" und "LaTeX => PS => PDF"
%% allow the following formats:
%%   .eps  .ps  .bmp  .pict  .pntg


%% Math Packages %%%%%%%%%%%%%%%%%%%%%%%%%%%%%%%%%%%%%%%%%%%%
\usepackage{amsmath,amsfonts,amstext,amscd,bezier,amsthm,amssymb}

\newcommand{\vect}[1]{\boldsymbol{#1}}

%\newenvironment{solu}{\noindent {\bf Solution.}\small }{\hfill $\square$ \normalsize \medskip}

%\renewcommand\thesection{\arabic{section}}
%\renewcommand*\thesection{Question \arabic{section}}

\def\MYTITLE{Review on boosting algorithms}

\title{\MYTITLE}
\author{\textsc{Vu} Tuan Hung and \textsc{Do} Quoc Khanh}
\date{\today}

\usepackage{fancyhdr}

\fancyhead[L]{\slshape Tuan Hung VU and Quoc Khanh DO}%\slshape\thepage LE,RO
\fancyhead[R]{\slshape Final report}%{\slshape \leftmark}
%\fancyhead[LO]{ccc}%{\slshape \rightmark}
%\fancyfoot[LO,LE]{}%\slshape Short Course on Asymptotics
\fancyfoot[C]{\thepage}
%\fancyfoot[RO,RE]{}%\slshape 7/15/2002

\DeclareMathOperator*{\argmax}{arg\,max}
\DeclareMathOperator{\Tr}{Tr}
\DeclareMathOperator{\diag}{diag}
\newcommand{\abs}[1]{{\lvert #1 \rvert}}
\newcommand{\norm}[1]{{\lVert #1 \rVert}}

\begin{document}
\maketitle
\pagestyle{fancy}

\section{Introduction}

\section{Two-class classification}
In this first part, we present an overview on boosting methods in the two-class classification framework. From a \textsl{training} set $(\textbf{x}_i, y_i)_{i = 1,...,N}$ in which $\textbf{x}_i \in \mathcal{X}$ and $y_i \in \mathcal{Y}$, we try to construct a function $F: \mathcal{X} \rightarrow \mathcal{Y}$ so that when a new value $\textbf{x}$ is randomly introduced, we have the highest probability to predict correctly the value $y$ corresponding to this value of $\textbf{x}$. Formally, we want to minimize the probability:
\begin{align}
    \mathbb{P}_{(\textbf{x}, y)} \left( y \neq F(\textbf{x})\right) \notag
\end{align}
The variable $\textbf{x}$ is called explanatory variables ($\textbf{x}$ may be multi-variational) and $y$ is called response variable. In the two-class classification framework, $\mathcal{Y} = \{ -1, 1\}$.

\subsection{Boosting and optimization in function space}
We exploit the point of view presented in \cite{trebst} by considering this problem as an estimation and optimization in function space. Indeed, if there exists a function $F^{*}$ which minimizes the above error:
\begin{align}
    F^{*} &= arg\min\limits_{F} \mathbb{P}_{(\textbf{x},y)} (y \neq F(\textbf{x})) \notag \\
    &= arg\min\limits_{F} \mathbb{E}_{(\textbf{x}, y)} \left[ 1(y \neq F(\textbf{x}))\right] \notag
\end{align}
then we are trying to estimate $F^{*}$ by a function $\hat{F}$ through the training set $(\textbf{x}_i, y_i)_{i=1,...,N}$.

\textbf{Base classifiers.} An approach frequently employed by classification algorithms is to suppose $F^{*}$ belongs to a function class parameterized by $\theta \in \Theta$:
\begin{align}
    F^{*} \in \mathcal{Q} = \left\lbrace F(., \theta) \vert \theta \in \Theta\right\rbrace \notag
\end{align}
so that the problem of estimating $F^{*}$ becomes an optimization of the parameters on $\Theta$:
\begin{align}
    \hat{\theta} = arg\min\limits_{\theta \in \Theta} \mathbb{E}_{(\textbf{x}, y)} \left[ 1(y \neq F(\textbf{x}, \theta))\right] \notag
\end{align}
and then we will take $\hat{F} = F(., \hat{\theta}) \in \mathcal{Q}$. For example, with regression tree algorithms, we have:
\begin{align}
    \mathcal{Q} = \left\lbrace F(x, \theta) = \sum\limits_{k=1}^K \lambda_k 1(\textbf{x} \in R_k) \vert (\lambda_1,...,\lambda_K) \in \mathbb{R}^K, (R_1,...,R_K) \in \mathcal{P}_{\mathcal{X}}\right\rbrace \notag
\end{align}
in which $\theta = (\lambda_{1:K}, R_{1:K})$ and $\mathcal{P}_{\mathcal{X}}$ is the set of all partitions of $\mathcal{X}$ into $K$ disjoint subsets by hyperplans which are orthogonal to axes. Similarly for support vector machines, $K$ disjoint subsets $R_1,...,R_K$ are divided by hyperplans in the reproducing kernel Hilbert space of $\mathcal{X}$ corresponding to some kernel.

We can see that a classifier is characterized by its function sub-space $\mathcal{Q}$ and the corresponding parameter space. Having the base classifiers $\mathcal{Q}_{1:M}$ with parameter spaces $\Theta_{1:M}$, instead of considering each of these classifiers separately, boosting methods consider functions of the following additive form:
\begin{align}
    \hat{F} \in \mathcal{F}_{\mathcal{Q}_1,...,\mathcal{Q}_M} = \left\lbrace \sum\limits_{m=1}^M \beta_m F(., \theta_m) \vert \theta_m \in \Theta_m, \forall m = 1,...,M\right\rbrace \notag
\end{align}
so that the optimization problem becomes:
\begin{align}
    \left\lbrace \hat{\beta}_{1:M}, \hat{\theta}_{1:M}\right\rbrace = arg\min\limits_{\beta \in \mathbb{R}^M, \theta_{1:M} \in \Theta_{1:M}} \mathbb{E}_{(\textbf{x}, y)} \left[ 1(y \neq F(\textbf{x}; \beta_{1:M}, \theta_{1:M}))\right] \label{opt_add}
\end{align}

Friedman, J. and Hastie, T. in \cite{boost} explained boosting as a forward stepwise algorithm for resolve the optimization problem \eqref{opt_add}. Friedman, J. in \cite{trebst} considered boosting like optimization algorithm in function space. We will try to adopt the latter to explain all mentioned boosting algorithms. Before going into greater details, we remark that as an classification algorithm, the boosting algorithm has its corresponding function subset which is $\mathcal{F} = \mathcal{F}_{\mathcal{Q}_1,...,\mathcal{Q}_M} = \sum\limits_{m=1}^M \mathcal{Q}_m$ so much larger than function subset of all base classifiers, explaining the dominating performance of boosting compared to its base classifiers.

\textbf{Loss function.} We remark that the binary loss function $1(y \neq F(\textbf{x}))$ is not the only function that reflects the difference between $y$ and $F(\textbf{x})$. In machine learning, one has other loss functions that are continuous, convex and then easier to do the optimization. For the rest of the report, we use in general $L(y, F(\textbf{x}))$ to indicate this function.

\textbf{Optimization on training set.} One difficulty is that we can not evaluate the distribution of $(\textbf{x}, y)$ and calculate the expectation in the right hand side formula of \eqref{opt_add}. Instead, we only want to optimize on the training data, which means the following optimization problem:
\begin{align}
    F^{*} = arg\min\limits_{F} \sum\limits_{i=1}^N L(y_i, F(x_i)) \notag
\end{align}
Put $Q(F) = \sum\limits_{i=1}^N L(y_i, F(\textbf{x}_i))$, we remark that $Q(F)$ depends only on $N$ values of the function $F$ at $(\textbf{x}_1,...,\textbf{x}_N)$. We denote for each function $F$ in the function space, a corresponding vector $\overline{F} \in \mathbb{R}^N$ so that $\overline{F} = (F(\textbf{x}_1),...,F(\textbf{x}_N))$, and we consider the relaxation problem on vector space $\mathbb{R}^N$: $\overline{F^{*}} = arg\min\limits_{\overline{F} \in \mathbb{R}^N} Q(\overline{F})$. We try to resolve this problem by recursive numerical methods. Suppose that at ${m-1}^{th}$ step we obtain a value $\overline{F}_{m-1}$. By numerical methods (Newton-Raphson, algorithm of gradient descent etc.) we find a direction of descent $d_m \in \mathbb{R}^N$ and a coefficient $c_m$ so that if we put $\overline{F}_m = \overline{F}_{m-1} + c_m.d_m$, then $Q(\overline{F}_m) \leq Q(\overline{F}_{m-1})$. But we can not use the direction $d_m$ directly in the original problem with functions $F_m$ because $\overline{F}_m$ identifies the values of functions only at $N$ points. Instead, we have to find a regression function near to the direction $d_m$; it means if we use a function subspace $\mathcal{Q}_m$ at $m^{th}$ step, then we have to solve:
\begin{align}
    \left\lbrace f_m, c\right\rbrace = arg\min\limits_{f_m\in \mathcal{Q}_m, c\in \mathbb{R}} \| d_m - c.\overline{f_m}\|^2 \notag
\end{align}
in which $\overline{f_m}$ is the vector of values of $f_m$ at $(\textbf{x}_1,...,\textbf{x}_N)$. After that, we have to look for a coefficient $\beta_m$ so that, if $F_{m-1}$ is the function obtained at the precedent step, then $Q(F_m) \leq Q(F_{m-1})$ with $F_m = F_{m-1} + \beta_mf_m$. If we start with $F_0 = 0$ then we obtain at $M^{th}$ the additive form $F_M = \sum\limits_{m=1}^M \beta_mf_m \in \mathcal{F} = \sum\limits_{m=1}^M \mathcal{Q}_m$. We summary this generic algorithm in the following table.
\begin{center}
	\fbox{
	    \parbox{0.9\textwidth}{
	        \textbf{Algorithm 1: Generic algorithm of boosting}\\
	        1. Start with $F_0(\textbf{x}) = 0$.\\
	        2. Repeat for $m = 1,2,...,M$:\\
	        (a) Search for a direction descent $d_m \in \mathbb{R}^N$ by some Newton-like numerical algorithm of optimization in $\mathbb{R}^N$.\\
	        (b) Solve $\left\lbrace f_m, c\right\rbrace = arg\min\limits_{f_m\in \mathcal{Q}_m, c\in \mathbb{R}} \| d_m - c.\overline{f_m}\|^2$.\\
	        (c) Search for a coefficient $\beta_m$ so that $Q(F_{m-1} + \beta_mf_m) \leq Q(F_{m-1})$, a line-search strategy can be used.\\
	        (d) $F_m = F_{m-1} + \beta_mf_m$.\\
	        3. Conclude with $F(\textbf{x})$.
	    }
	}
\end{center}

\subsection{One-degree optimization}

\subsection{Two-degree optimization}

\section{Multi-class classification and some generalizations}

\subsection{A traditional approach}

\subsection{Some generalization of two-class algorithms}

\subsection{Other generalizations}

\section{Experiments}

\subsection{Experiments with simulated data}

\subsection{Experiments with real data}

\section{Conclusion}

\begin{thebibliography}{9999}%\enlargethispage{\baselineskip}
\bibitem[1]{boost}Friedman, J., Hastie, T. \& Tibshirani, R. \textsl{Additive Logistic Regression: a Statistical View of Boosting}, 2000.

\bibitem[2]{trebst}Friedman, J. \textsl{Greedy Function Approximation: A Gradient Boosting Machine}, IMS 1999 Reitz Lecture, 2001.

\bibitem[3]{SchaAndSin1998}Schapire, R.E. \& Singer, Y. \textsl{Improved Boosting Algorithms: Using Confidence-rated Predictions}, 1998.

\end{thebibliography}

\end{document}